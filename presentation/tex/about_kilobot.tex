\section{Introduction}
\begin{frame}
	\frametitle{Kilobots}
	\begin{columns}
		\begin{column}{0.5\textwidth}
			\textbf{Specifications}
			\vspace{0.2cm}
			\begin{itemize}
				\item ATmega 328p processor 
				\item Li-Ion 3.7V battery 
				\item One IR transmitter-receiver pair 
				\item One light sensor 
				\item Two vibration motors (1 cm/sec, 45 degrees/sec)
			\end{itemize}
		\end{column}
	\begin{column}{0.5\textwidth}
		\begin{figure}
			\centering
			\fbox{\includegraphics[scale=0.03]{kilobots}}
			\hspace{5cm}
			\caption{Kilobot}
		\end{figure}
	\end{column}
	\end{columns}
\end{frame}

\begin{frame}
	\frametitle{About Kilobots \cite{kilobotics_manual}}
	\begin{columns}
		\begin{column}{0.5\textwidth}
			\begin{figure}
				\centering
				\fbox{\includegraphics[scale=0.35]{comm}}
				\hspace{0.2cm}
				\caption{Communication between two Kilobots}
			\end{figure}
		\end{column}
		\begin{column}{0.5\textwidth}
			\begin{itemize}
				\item Reflecting IR light
				\item Communication up to 7 cm (32kb/s) away 
				\item Using over-head controller
			\end{itemize}
		\end{column}
	\end{columns}
\end{frame}

\begin{frame}
	\frametitle{Wireless communication in Kilobots}
	\begin{itemize}
		\item Sharing of same wireless channel by all robots
		\item CSMA-CA (Carrier Sense Multiple Access with Collision Avoidance) method \cite{WEBOPEDIA-csma-cd}.
		\item Reduction of channel bandwidth        
	\end{itemize}
\end{frame}
